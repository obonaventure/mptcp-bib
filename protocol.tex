
The first phase of the work in the IETF MPTCP working group has been focussed on the production of several documents. The architectural guidelines are specified in \cite{rfc6182}. This document served as a reference for the work in the IETF working group. The main design decision was that Multipath TCP assumes that the communicating hosts have different addresses and that these addresses are used to identify the flows. The main requirements listed in this document were :
\begin{itemize}

\item \emph{Improve Throughput}: Multipath TCP MUST support the concurrent use of multiple paths.  To meet the minimum performance incentives for deployment, a Multipath TCP connection over multiple paths SHOULD achieve no worse throughput than a single TCP connection over the best constituent path.
\item \emph{Improve Resilience}: Multipath TCP MUST support the use of multiple paths interchangeably for resilience purposes, by permitting segments to be sent and re-sent on any available path. 
\end{itemize}

In addition, \cite{rfc6182} lists several compatibility goals that include compatibility with existing applications that use the socket API and compatibility with the network including the middleboxes that it could contain.

The security requirements were discussed in more details in a separate document \cite{rfc6181}. These requirements are being updated in \cite{draft-ietf-mptcp-attacks}.



The detailed specification for the Multipath TCP protocol extension may be found in \cite{rfc6824}. This specification is being updated to include in the revised specification the lessons learned from the utilisation of Multipath TCP in the global Internet \cite{draft-ietf-mptcp-experience}. Several of the design choices for the protocol and its implementation in the Linux kernel are discussed in \cite{Raiciu_Hard:2012}. The revised specification \cite{draft-ietf-mptcp-rfc6824bis} includes several improvements such as a modified \texttt{ADD\_ADDR} option that includes a HMAC and an option for the RST segments \cite{draft-bonaventure-mptcp-rst}.

The standard congestion congestion scheme chosen by the MPTCP working group for Multipath TCP is described in \cite{rfc6356}. This congestion control scheme was first proposed and evaluated in \cite{Wischik_Design:2011}.

One of the initial design choices of Multipath TCP was to be as compatible as possible with the existing applications that interact with TCP through the socket interface. Still, \cite{rfc6897} describes some basic extensions to the socket API to enable applications to notably disable Multipath TCP.  


\subsection{Protocol extensions}



\subsection{Interactions with other protocols}

In several use cases, such as datacenters, the selection of the best paths for the subflows that compose a Multipath TCP connection is an important decision. The current IETF RFCs do not manadate any path selection mechanisms. Proposed techniques include random selection with ECMP \cite{Raiciu_Datacenter:2011}, modifying the hash function used by ECMP \cite{Detal_Revisiting:2013} is used, or using other fields like the TTL to influence the path as proposed in \cite{Kabbani_Flowbender:2014}. Another approach is to develop a signalling protocol that enables the hosts to query the network for the paths to be used to reach a specific destination. Krupakaran et al. propose in \cite{Krupakaran_Optimized:2015} a layer-2 protocol called Traceflow to exchange this kind of information.

